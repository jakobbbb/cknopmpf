\documentclass[notheorems]{beamer}

\usepackage{lipsum}

\usepackage[
  %german,
  %light
]{cknopmpf}


\title{Hello, World!}
\subtitle{A Presentation about Things}
\author{Alice Lastname\inst{1} \and Bob Nachname\inst{2}}
\shortauthor{Lastname, Nachname}
\email[1]{alice@example.org}
\email[2]{bob@example.org}


%-----------------------------------------------------------------------


\begin{document}
\maketitle

\section{Lorem Ipsum}

\begin{frame}{A Paragraph of Lipsum}
  \lipsum[1]
\end{frame}

\begin{frame}{Dolor Sit}
\textbf{Consectetur adipiscing}:
  \begin{itemize}
    \item eiusmod
    \item tempor
    \item incididunt
  \end{itemize}

\textbf{Adipiscing pariatur}:
  \begin{enumerate}
    \item eiusmod
    \item tempor
    \item incididunt
  \end{enumerate}
\end{frame}


\begin{frame}
  \frametitle{A Proof}
  \framesubtitle{Oh yeah}
  \begin{thm}
    \begin{equation*}
      \sum_{i=1}^n i = \frac{n(n+1)}{2},%
      \quad\forall n\in\mathbb{N}.
    \end{equation*}
  \end{thm}
  \begin{proof}\noqed
    Basis:
    \begin{equation*}
      \sum_{i=1}^1 i = 1 = \frac{1(1+1)}{2}.
    \end{equation*}
  \end{proof}
\end{frame}
\begin{frame}
  \begin{proof}
    Suppose the statement holds for an
    $n\in\mathbb{N}$.

    Induction:
    We'll show that, as $n\longrightarrow n+1$:
    \begin{equation*}
      \sum_{i=1}^{n+1} i = \frac{(n+1)(n+2)}{2}
      = \frac{n^2 + 3n + 2}{2}.
    \end{equation*}
    Per the induction hypothesis, $(1)$ holds, and thus:
    \begin{align*}
      \sum_{i=1}^{n+1} i
      &= \sum_{i=1}^{n} i + (n+1)
      \overset{(1)}{=} \frac{n(n+1)}{2} + (n+1) \\
      &= \frac{n(n+1)}{2} + \frac{2(n+1)}{2}
      = \frac{n^2 + 3n + 2}{2}.
    \end{align*}
  \end{proof}
\end{frame}

\end{document}
